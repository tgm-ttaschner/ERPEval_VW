\documentclass[12pt]{article}

\usepackage[english]{babel}
\usepackage[utf8x]{inputenc}
\usepackage{amsmath}
\usepackage{enumitem}
\usepackage{graphicx}
\usepackage{ulem}
\usepackage{caption}
\usepackage{placeins}
\usepackage[usenames,dvipsnames]{color}
\usepackage[colorinlistoftodos]{todonotes}
\usepackage{listings}
\usepackage{fixltx2e}
\usepackage{scrpage2}
\usepackage{lastpage}
\clearscrheadfoot
\pagestyle{scrheadings}
\usepackage{glossaries}
\usepackage[
top    = 2.75cm,
bottom = 2.00cm,
left   = 2.50cm,
right  = 2.00cm]{geometry}
\setcounter{secnumdepth}{4}


\makeglossaries

\newglossaryentry{erp} {name=ERP, description={Enterprise Resource Planning}}
\newglossaryentry{glossaryVerweis} {name=abkuerzung, description={Langer Name}}


\begin{document}
	\begin{titlepage}
		\begin{center}
			% Oberer Teil der Titelseite:
			\includegraphics[width=0.5\textwidth]{images/logo}\\[1cm]    
			
			\LARGE TGM - HTBLuVA Wien XX \\ IT Department  \\[1.5cm]
			
			% Title
			\rule{1.0\textwidth}{1mm}
			{ \huge \bfseries \\[0.4cm]  \huge Volkswagen AG \\ \LARGE ERP-Evaluation \\[0.4cm] }
			
			\rule{1.0\textwidth}{1mm}
			
			
			
			% Author and supervisor
			\noindent 
			\vspace{5cm}
			
			\begin{center}
				\large
				Authors: \\ 
				Bergler \textsc{Adrian} \&
				Haidn \textsc{Martin} \&
				Siegel \textsc{Hannah} \&
				Soyka \textsc{Wolfram}
			\end{center}
			
			\vfill
			
			% Bottom of the page
			{\large \today}
			
		\end{center}
	\end{titlepage}
	
	\tableofcontents
	
	
	%HEADER AND FOOTER
	\pagenumbering{arabic}
	\ohead{\headmark}
	\automark{section}
	\ifoot{© Haidn, Siegel, Soyka}
	\ofoot{\pagemark ~of \pageref{LastPage}}
	
	\newpage
	
	\section{Auswahl des Unternehmens}
	
	\section{Volkswagen AG}
	http://www.economist.com/node/21558269
	\subsection{Das Unternehmen}
	Die Volkswagen AG, ist Europas größter Automobilhersteller. Zum Volkswagen-Konzern gehören die Marken Audi, Bentley, Bugatti, Lamborghini, Seat, Skoda, Volkswagen und Volkswagen Nutzfahrzeuge. Allein in Deutschland gibt es neun Volkswagenwerke.
	
	\subsubsection{Historie}
	Der Volkswagen - die Idee, die relativ neue Erfindung der Automobilität in Form eines Wagens der für den kleinen Mann bezahlbar sei an selbigen zu bringen - kam erstmals im um 1904 herum auf. Allerdings vergingen 33 Jahre bis im Mai 1937 dazu dann die „Gesellschaft zur Vorbereitung des Deutschen Volkswagen mbH“ gegründet, welche im Jahr darauf in „Volkswagenwerk GmbH“ umbenannt wurde.\cite{vwchronik} \\
	In Wolfsburg wurde das VW-Werk errichtet, wo der sogenannte KdF-Wagen (Kraft durch Freude), der größtenteils vom Konstrukteur Ferdinand Porsche konzipiert war, hergestellt wurde. Von Anfang an war den Herstellern allerdings aus vorhergehenden Versuchen diverser Automobilherstellern klar, dass der von Hitler geforderte Preis von höchstens 1000 Reichsmark nicht eingehalten werden konnte. \cite{geschdautos}\\
	Während des 2.Weltkrieges wurde aufgrund von neuen Prioritäten und Ressourcenmangel die Herstellung von Autos im VW Werk größtenteils gestoppt und stattdessen auf Rüstungsgüter umgestellt, unter anderem auch auf Produktion der Vergeltungswaffe V1.\cite{autowp} Während dem 2. Weltkrieg wurde außerdem das KZ Arbeitsdorf nahe dem VW Werk erbaut, um das Werk mit Arbeitskräften zu versorgen.\cite{terror}  
	\\
	Nach dem Krieg werden im VW Werk (wieder) Autos gebaut. Innerhalb von 10 Jahren schafft es VW nicht nur das Werk von Kriegsschäden zu reparieren, sondern auch eine Million Käfer zu produzieren. \cite{ahwest}\\
	Am 22. August 1960 wurde aus der "Volkswagen GmbH" eine Aktiengesellschaft, neun Jahre Später übernahm VW die Auto Union GmbH, der die Marke Audi gehörte. Seit diesem Zeitpunkt hat der Volkswagen Konzern mehr als eine Marke Autos im Angebot, was sich im laufe der nächsten Jahre auf stolze 12 Marken erhöhte (Stand Dez. 2013). \cite{vwag}
	Die nächsten zwei Jahrzehnte war es vergleichsweise Ruhig um  Volkswagen, der in dieser Zeit allerdings keineswegs untätig war sondern u.a. das Erfolgsauto Golf und Passat hervorbrach und weitere Marken akquirierte, bis der Konzern im Geschäftsjahr 2003 einen Gewinneinbruch von 50 \% erlitt. Gründe hierfür waren laut VW notwendige Restrukturierungen in Brasilien, eine Rekordzahl an neuen Modellen und eine schlechte Situation am Weltmarkt. \cite{sud} \\
	Zwei Jahre später hatte sich der Konzern zwar halbwegs von den Gewinneinbußen erholt, allerdings schrieb er wieder schlechte Schlagzeilen mit einem Korruptionsskandal und Umfassenden Streiks in einem Brasilianischen Werk. \cite{autowp}\\
	2014 schaffte es VW mit 10,14 Millionen verkauften Fahrzeugen einmal mehr nur auf den zweiten Platz der größten Automobilhersteller der Welt, knapp hinter Toyota mit 10,23 Millionen verkauften Fahrzeugen. Das Toyota 2015 allerdings mit Einbußen im Heimatland Japan rechnet könnte es VW 2015 schaffen erstmals der größte Autohersteller der Welt zu werden.
	
	
	
	
	\newpage
	\listoftables
	\listoffigures
	\printglossaries
	\subsection{Easy Bibliography}
	\begin{thebibliography}{56}
		
		
		
		\bibitem{geschdautos} 
		\textbf{Geschichte des Autos} \\
		\textit{
			http://www.fundus.org/referat.asp?ID=11034
		}
		\newline last used: 27.03.2015, 14:23
		
		
		\bibitem{autowp} 
		\textbf{Entstehungsgeschichte von VW} \\
		\textit{
			http://www.autowallpaper.de/Wallpaper/VW/Entstehungsgeschichte\_VW.htm
		}
		\newline last used: 27.03.2015, 14:26
		
		\bibitem{terror} 
		\textbf{Wolfgang Benz, Barbara Distel}, 2005-2009 \\
		\textit{
			Der Ort des Terrors. Geschichte der nationalsozialistischen Konzentrationslager
		}
		
		
		\bibitem{ahwest} 
		\textbf{Volgswagen AG Geschichte} \\
		\textit{
			http://www.autohaus-westend.de/volkswagen-ag-geschichte/
		}
		\newline last used: 27.03.2015, 13:41
		
		\bibitem{vwag} 
		\textbf{Volgswagen AG} \\
		\textit{
			http://www.volkswagenag.com/
		}
		\newline last used: 27.03.2015, 9:13
		
		\bibitem{sud} 
		\textbf{Drastische Gewinneinbruch bei VW} \\
		\textit{
			http://www.sueddeutsche.de/wirtschaft/geschaeftsjahr-drastischer-gewinneinbruch-bei-vw-1.814556
		}
		\newline last used: 27.03.2015, 12:11
		
		\bibitem{vwchronik} 
		\textbf{VW Chronik} \\
		\textit{
http://www.chronik.volkswagenag.com/
		}
		\newline last used: 29.03.2015, 12:32
		

		
		
		
	\end{thebibliography}
\end{document}
