\documentclass[12pt]{article}

\usepackage[english]{babel}
\usepackage[utf8x]{inputenc}
\usepackage{amsmath}
\usepackage{enumitem}
\usepackage{graphicx}
\usepackage{ulem}
\usepackage{caption}
\usepackage{placeins}
\usepackage[usenames,dvipsnames]{color}
\usepackage[colorinlistoftodos]{todonotes}
\usepackage{listings}
\usepackage{fixltx2e}
\usepackage{scrpage2}
\usepackage{lastpage}
\usepackage{glossaries}

\clearscrheadfoot
\pagestyle{scrheadings}
\usepackage[
top    = 2.75cm,
bottom = 2.75cm,
left   = 2.50cm,
right  = 2.00cm]{geometry}
\setcounter{secnumdepth}{4}

\begin{document}
\begin{titlepage}
\begin{center}
% Oberer Teil der Titelseite:
\includegraphics[width=0.9\textwidth]{images/vwlogo}\\[1cm]    


% Title
\rule{1.0\textwidth}{1mm}
{ \huge \bfseries \\[0.4cm]  \huge ERP-Evaluation \\[0.4cm] }
\LARGE TGM Wien XX - Höhere Abteilung für Informationstechnologie  \\[0.4cm]

\rule{1.0\textwidth}{1mm}




% Author and supervisor
\noindent 
\vspace{3cm}

\begin{center}
\large
Hagen Aad Fock \&
Stefan Polydor \&
Thomas Taschner \&
Michael Weinberger
\end{center}

\vfill

% Bottom of the page
{\large \today}

\end{center}
\end{titlepage}

\tableofcontents


%HEADER AND FOOTER
\pagenumbering{arabic}
\ohead{\headmark}
\automark{section}
\ifoot{© Bergler, Haidn, Siegel, Soyka}
\ofoot{\pagemark ~of \pageref{LastPage}}

\newpage

\noindent
Wir haben uns für das Unternehmen Volkswagen Group entschieden, da es ein produzierenden Unternehmen im deutschsprachigem Raum ist. Da es die Gesellschaftsform einer Aktien Gesellschaft hat, sind die Zahlen sowie sämtliche Informationen einsichtlich. 
\section{Volkswagen AG}
\subsection{Das Unternehmen}
Die Volkswagen AG ist Europas größter Automobilhersteller. Zum Volkswagen-Konzern gehören die Marken Audi, Bentley, Bugatti, Lamborghini, Seat, Skoda, Volkswagen und Volkswagen Nutzfahrzeuge. Allein in Deutschland gibt es neun Volkswagenwerke.
\subsubsection{Historie}
Der Volkswagen - die Idee, die relativ neue Erfindung der Automobilität in Form eines Wagens der für den kleinen Mann bezahlbar sei an selbigen zu bringen - kam erstmals im um 1904 herum auf. Allerdings vergingen 33 Jahre bis im Mai 1937 dazu dann die „Gesellschaft zur Vorbereitung des Deutschen Volkswagen mbH“ gegründet, welche im Jahr darauf in „Volkswagenwerk GmbH“ umbenannt wurde.\cite{vwchronik} \\
In Wolfsburg wurde das VW-Werk errichtet, wo der sogenannte KdF-Wagen (Kraft durch Freude), der größtenteils vom Konstrukteur Ferdinand Porsche konzipiert war, hergestellt wurde. Von Anfang an war den Herstellern allerdings aus vorhergehenden Versuchen diverser Automobilherstellern klar, dass der von Hitler geforderte Preis von höchstens 1000 Reichsmark nicht eingehalten werden konnte. \cite{geschdautos}\\
Während des 2.Weltkrieges wurde aufgrund von neuen Prioritäten und Ressourcenmangel die Herstellung von Autos im VW Werk größtenteils gestoppt und stattdessen auf Rüstungsgüter umgestellt, unter anderem auch auf Produktion der Vergeltungswaffe V1.\cite{autowp} Während dem 2. Weltkrieg wurde außerdem das KZ Arbeitsdorf nahe dem VW Werk erbaut, um das Werk mit Arbeitskräften zu versorgen.\cite{terror}  
\\
Nach dem Krieg werden im VW Werk (wieder) Autos gebaut. Innerhalb von 10 Jahren schafft es VW nicht nur das Werk von Kriegsschäden zu reparieren, sondern auch eine Million Käfer zu produzieren. \cite{ahwest}\\
Am 22. August 1960 wurde aus der "Volkswagen GmbH" eine Aktiengesellschaft, neun Jahre Später übernahm VW die Auto Union GmbH, der die Marke Audi gehörte. Seit diesem Zeitpunkt hat der Volkswagen Konzern mehr als eine Marke Autos im Angebot, was sich im laufe der nächsten Jahre auf stolze 12 Marken erhöhte (Stand Dez. 2013). \cite{vwag}
Die nächsten zwei Jahrzehnte war es vergleichsweise Ruhig um  Volkswagen, der in dieser Zeit allerdings keineswegs untätig war sondern u.a. das Erfolgsauto Golf und Passat hervorbrach und weitere Marken akquirierte, bis der Konzern im Geschäftsjahr 2003 einen Gewinneinbruch von 50 \% erlitt. Gründe hierfür waren laut VW notwendige Restrukturierungen in Brasilien, eine Rekordzahl an neuen Modellen und eine schlechte Situation am Weltmarkt. \cite{sud} \\
Zwei Jahre später hatte sich der Konzern zwar halbwegs von den Gewinneinbußen erholt, allerdings schrieb er wieder schlechte Schlagzeilen mit einem Korruptionsskandal und Umfassenden Streiks in einem Brasilianischen Werk. \cite{autowp}\\
2014 schaffte es VW mit 10,14 Millionen verkauften Fahrzeugen einmal mehr nur auf den zweiten Platz der größten Automobilhersteller der Welt, knapp hinter Toyota mit 10,23 Millionen verkauften Fahrzeugen. Das Toyota 2015 allerdings mit Einbußen im Heimatland Japan rechnet könnte es VW 2015 schaffen erstmals der größte Autohersteller der Welt zu werden.

\subsection{Finanzen}
\textbf{Umsatz}\\
Im Jahr 2013 lag der Umsatzerlös bei 65.587 Millionen Euro. Im Jahr zuvor jedoch betrug dieser 68.361 Millionen Euro. Ein Rückgang von 226 Millionen Euro ist dadurch entstanden. Gleichzeitig ist auch ein Gewinnrückgang von mehr als 3 Milliarden Euro festzuhalten. Gründe dafür lassen sich wie folgt finden:
\\
"Die in Vorjahren erworbenen 73,7\% der Anteile am Grundkapital der MAN SE, München, (9,1 Mrd.Euro) wurden von
der Volkswagen AG im Geschäftsjahr in die Truck \& Bus GmbH, eine 100-prozentige Tochtergesellschaft eingebracht.
Zusätzlich hat die Volkswagen AG 3,3 Mrd.Euro in die Kapitalrücklage der Truck \& Bus GmbH eingezahlt. Von der Truck \&
Bus GmbH wurden 2013 insgesamt 1,0 Mrd. Euro Verluste aufgrund des Beherrschungs- und Gewinnabführungsvertrags
mit der MAN SE übernommen." \cite[Seite 3]{jbilanz2013vw}
\\\\
"Die Volkswagen AG hat von der Volkswagen Bank GmbH, Braunschweig, eine Beteiligung erworben und diese anschließend im Wege der Sacheinlage (1,7 Mrd.Euro) in die VW Finance Luxemburg S.A., Luxemburg, eingebracht.\\
Darüber hinaus wurden Kapitalzuführungen bei der AUDI AG, Ingolstadt, (1,9 Mrd.Euro) und kleinere Kapitalmaßnahmen bei verbundenen Unternehmen durchgeführt. Bei der Global Automotive C.V. Amsterdam, Niederlande
wurde eine Sachkapitalherabsetzung in Höhe von 1,1 Mrd.Euro durchgeführt. Die Volkswagen AG hat im HI-TV Fonds (TreasuryFonds) 1,0 Mrd.€ angelegt. "\cite[Seite 4]{jbilanz2013vw}
\\ \\ 
\textbf{Aktie} \\
Am 7. April 1961 war das IPO von Volkswagen. Derzeit liegt das totale Marktkapital der Stamm- und Vorzugsaktien bei 114,71 Mrd Euro. %TODO EUR

\begin{figure}[!h]
\centering
\includegraphics[width=0.7\textwidth]{images/finanzen2015}
\caption{Zehn Jahres Übersicht der VW-Vorzugsaktie \cite{aktienfotos}}
\label{fig:vwaktie1}
\end{figure}\FloatBarrier
\noindent
Wie in Abbildung \ref{fig:vwaktie1} zu sehen ist, ist die Volkswagen Vorzugsaktie in den letzten zehn Jahren stetig gestiegen. In 2008 wurde der Fall der Aktie durch die Finanzkrise bedingt.\\
Der Stand am Dienstag, dem 31. März 2015, der VW Vorzugsaktie (Xetra) war bei 249,90 Euro.
\begin{figure}[!h]
\centering
\includegraphics[width=0.7\textwidth]{images/finanzen20151}
\caption{Jahres Übersicht der VW-Aktie \cite{aktienfotos}}
\label{fig:vwaktie3}
\end{figure}\FloatBarrier
\noindent
Im letzten Jahr war das Minimum im Oktober bei 160 Euro erreicht. Seit diesem Tag ist sie auch stetig bis auf kleine Ausnahmen gestiegen. \\\\
\textbf{Aktionärsstruktur}\\
Mit 31.12.2014 waren insgesamt 180.641.478 Vorzugsaktien und 295.089.818 Stammaktien ausstehend.\\
Im Juni 2014 hat die Volkswagen Aktiengesellschaft 10.471.204 neue Vorzugsaktien ausgegeben. Zusätzlich wurden im 1. Halbjahr 2014 22.103 Vorzugsaktien aus der Wandlung von Pflichtwandelanleihen geschaffen.
\cite{aktionaersstruktur} \\ \\
\textit{Übersicht über die Stimmrechtsverteilung} \\
50,73\% Porsche Automobil Holding SE, Stuttgart\\
20,00\% Land Niedersachsen, Hannover\\
17,00\% Qatar Holding LLC\\
12,30\% Weitere
\begin{table}
\begin{tabular}{|p{0.75\textwidth}|p{0.25\textwidth}|}
\hline

Die wichtigsten Zahlen hier noch einmal zusammengefasst:  & \cite{yahoofinanzenvw} \\  \hline
\textbf{Geschäftsjahr 2014}  & = Kalenderjahr \\  \hline
\textbf{Rentabilität}  & \\  \hline
 Gewinnspanne &   5,89\% \\ \hline
 \textbf{Managementeffektivität}  & \\  \hline
Kapitalrentabilität  & 2,21\%  \\  
Eigenkapitalrendite  &   12,28\%  \\  \hline
 \textbf{GuV}  & \\  \hline

 Umsatz &   202,46Mrd. \\  
 Umsatz pro Aktie &  408,12  \\  
 
Vierteljährliches Umsatzwachstum &   6,60\% \\  
 Bruttoergebnis vom Umsatz &34,39Mrd.    \\  
EBITDA  &  20,50Mrd.  \\  
 Auf Stammaktien entfallender Jahresüberschuss  &  10,98Mrd.  \\  \hline
  \textbf{Bilanz}  & \\  \hline


Cash (gesamt) : &  	28,30Mrd.  \\  

Gesamt-Cash pro Aktie : & 59,50   \\  
Schulden (gesamt) : &   108,65Mrd. \\  
 Schulden/Equity (gesamt) :&  	120,47  \\  \hline
   \textbf{Cash Flow-Aufstellung}  & \\  \hline

Cash Flow aus betrieblichen Tätigkeiten  &  10,78Mrd.  \\  
Levered Free Cash Flow  &  8,45Mrd.  \\  \hline

\end{tabular}
\end{table}\FloatBarrier
\subsubsection{Übernahme von Porsche}

%\begin{figure}[!h]
%\centering
%\includegraphics[width=0.1\textwidth]{images/infografik-volkswagen-henkel}
%\includegraphics[width=0.1\textwidth]{images/test}

%\caption{Vergleich der VW-Stammaktie zu VW-Vorzugsaktien und zu Zeit der Finanzkrise. \cite{adsfhgterwdsfhgf2}}
%\label{fig:vwaktie2}
%\end{figure}\FloatBarrier
%\noindent
\begin{figure}[!h]
\centering
\includegraphics[width=0.7\textwidth]{images/finanzen2015S}
%\includegraphics[width=0.1\textwidth]{images/test}
\caption{VW-Stammaktie, 10 Jahres Übersicht}
%\caption{Vergleich der VW-Stammaktie zu VW-Vorzugsaktien und zu Zeit der Finanzkrise. \cite{adsfhgterwdsfhgf2}}
\label{fig:vwaktie2}
\end{figure}\FloatBarrier
\noindent
Im Oktober 2008 ist die Stammaktie von Volkswagen um 12 Milliarden gestiegen. Der Grund hierfür war, dass Porsche über drei Jahre hinweg einen Anteil an Volkswagen erworben hatte. \\
Die gesamte Geschichte ist sehr kompliziert, jedoch lässt sie sich vereinfacht wie folgt darstellen:\\\\
\textbf{2005}\\
Im Jahr 2005 begann die Firma Dr. Ing. h.c. F. Porsche AG Stammaktien der Firma VW zu kaufen. Bis Ende 2005 erwarben sie mit den liquiden Mitteln insgesamt 18\% der Stammaktien für etwa 3.5 Milliarden Euro. Porsche AG wollte in Volkswagen investieren, da sie selbst Angst für einer Übernahme hatten und den gross Konzern zu erst kontrollieren wollte. Die beiden Firmen teilten sich damals bereits  einige Produktionsschritte und waren allgemein sehr ähnlich. \\\\
\textbf{2006}\\
2006 steigerte Porsche den Anteil an VW auf 30\%.\\\\
\textbf{2007}\\
Die Porsche Holding SE, Stuttgart wird gegründet, in welche die Dr. Ing. h.c. F. Porsche AG  eingegliedert wird.\\ Die EU ändert ein Volkswagen internes Firmengesetz: Davor konnten Aktionäre nie mehr als 20\% Stimmrechte innehaben ganz egal wie gross ihr Anteil ist. Da dies gegen EU Regeln verstösst, wird dieses Gesetz abgeschafft.\\\\
\textbf{2008}\\
Porsche hat nun 35.14\% von Volkswagen inne, muss dafür jedoch einen Kredit von 6 Milliarden aufnehmen. \\
Dieser Kauf hat den Preis sehr hoch getrieben, weit über einen Wert, welcher wirtschaftlich Sinnvoll war. (Abbildung \ref{fig:vwaktie2}) Durch Spekulation wurden nun Aktien von sowohl Porsche als auch VW gekauft und verkauft. Am 26 Oktober gab Porsche an, dass es nun 43\% von VW besitzt und die Möglichkeit hätte noch mehr zu kaufen, ein Interesse bestand bis zu 75\% zu erwerben. \\
Daraufhin stieg der Preis der VW Aktien auf über 1000 Euro, und Volkswagen war für diese Zeit die Firma welche weltweit am Meisten wert war. Der Kauf von Volkswagen schlägt fehl und die Verantwortlichen, Wendelin Wiedeking und Holger Härter werden entlassen.
\\\\
\textbf{2009}\\
Im Jänner hat Porsche nun 52\% der Volkswagen Group AG Aktien inne und Schulden in Höhe von 10 Milliarden Euro.
Auf Grund der Schulden hat die Familien Porsche Abschied davon genommen, ihren Anteil an VW auf 75 Prozent aufzustocken um den Konzern kontrollieren zu können und eine Not-Fusion mit Volkswagen entsteht. \\
Am 13. August wird ein Vertag aufgesetzt, welcher den Vorstand von Volkswagen in den Aufsichtsrat der Porsche Holding SE erhebt. Des weiteren werden 49.9\% (um 3.9 Milliarden Euro) von der Dr. Ing. h.c. F. Porsche AG an Volkswagen verkauft. 10 \% der Porsche Holding SE gehen an Quatar Holding LLC. \\\\
\textbf{2011}\\
Eine Verschmelzung der Volkswagen Group und der Porsche Holding SE ist geplant, jedoch ist dies erst möglich sobald alle Klagen gegen Porsche Holding SE beendet sind. Die in den Jahren zuvor (insbesondere Oktober 2008) passierten Ereignisse hatten Hedgefund klagen zur Folge. Die Porsche Holding SE wird der Marktmanipulation und der Absprache bezichtigt.\\\\
\textbf{2012}\\
Wegen den noch immer hohen Schulden verkauft die Porsche Holding SE die Restlichen Anteile an der  Dr. Ing. h.c. F. Porsche AG and Volkswagen um 4.5 Milliarden Euro.
\cite{squeezy}

\newpage
\subsection{Produktspektrum}
Im Mittelpunkt der Produktion steht das Automobil und wird von eine Anzahl an vielseitigen Dienstleistungen rund um das Thema Fahren verstärkt. Im Finanziellen Sektor sind Dienstleistungen bereits auf eine eigene Gesellschaft "Volkswagen Financial Services" ausgelagert.
Das Produktspektrum der Volkswagen AG beinhaltet alle Kfz vom Stadtfahrzeug, über Motorad, bis zum Großtransporter.
So unterteilt sich der Konzern in die Folgenden Marken und Tochtergesellschafen.\\
\begin{figure}[!h]
\centering
\includegraphics[width=0.7\textwidth]{images/Volkswagen-Group-Brands}
\caption{Marken und Tochtergesellschaften der Volkswagen AG. \cite{marken}}
\end{figure}\FloatBarrier
\noindent
\textbf{Volkswagen}\\
Das Segment von Volkswagen ist überwiegend auf den Endverbaucher abgestimmt. Es beinhaltet den kleinen Stadtflitzer bis zum SUV. Die klingensten Modelle darunter sind unter anderem Polo, Golf, Passat und Sharan, Touran, Tigeran und Scirocco.
Auch im elektrifizierten Modellbereich ist die Marke mit der e-up! und e-golf Serie vertreten.\\
\\
\textbf{Volkswagen Commercial Vehicles}\\
Die zweite eigene Untermarke des Konzerns kosentriert sich auf Transport und Nutzfahrzeuge. Die Modellpalette Umfasst mitunter Caddy's, Van's, Hecklader und unterschiedliche Versionen von Transportern. Wie der Name schon verrät sind die Volkswagen Commercial Vehicles primär an die Bedürfnisse von Kunden zugeschnitten, die auf Grund ihrer Tätigkeiten Kraft und Stauraum benötigen.
\newpage
\textbf{Audi}\\
Die von den vier Ringen geprägte Audi AG, welche symbolisch für den 1932 stattgefundenen Zusammenschluss von
Audi, DKW, Horch und Wanderer steht, produziert heute in den gängigen Klassen A, S, T, Q und weitere Kleinserien in der typischen Buchstabenbenennung. Mit der tron technologie steht der Audi A3 Sportback auch in einer Hybrid-Version mit Benzin/Strom und Benzin/Erdgaß zur Verfügung.
Seit mittlerweile 100 Jahren der Produktion sind die Klassen in ihrer achten Generation angelangt und bietet seinen Fahrern eine weitgestreutet Modellpallette von Kleinwagen, über Sportwagen und SUV, bis zur Oberkalsse. Die Audi AG bietet, sowie die zwei folgenden VW Töchter Skoda und Seat die Möglichkeit für Unternehmen Business-Pakete an, mit dem Wagen für als Firmenfahrzeuge gekauft und geleased werden können.\\
\\
\textbf{Skoda}\\
Die Modelle der Marke Skoda sind eher in der Mittelklasse orientiert, bieten dem Fahrer allerdings zu günstigeren Preisen als manche andere Konzerntöchter, eine Vielzahl von aktuellen Features und neuen technologien zum Thema Fahrkomfort und Sicherheit. Die gängisten Modelle sind Fabia, Rapid und Octavia.\\
\\
\textbf{Seat}\\
Der spanische Autohersteller Seat ist im Kleinwagen und Mittelklasseberreich zu Hause. Der Konzern wurde 1950 mit der Unterstützung des spanischen natianal Institutes für Industry und Fiat ins leben gerufen, daher waren die ersten Modelle fast ausschließlich Lizenzbauten der Marke fiat. Bereits drei Jahre danach wurde das erste werkseigene Serienmodell produziert und 1965 begann der Export in andere Länder.
Die gängigsten Serien der Marke sind heute unter anderen Ibiza, Leon, Altea und Alhambra, welche hauptsächlich in der Mittelklasse angesiedelt sind.\\
\\ 
\textbf{Porsche}\\
Der Premiumhersteller Porsche, welcher ursprünglich plante Volkswagen zu übernehmen, ging auf Grund von großen Schulden eine Zwangs-Allianz mit Europas größtem Autokonzern ein und wurde so schlussendlich doch ein Teil der Volkswagen Group.
Heute liegt er mit 18,4\% Gewinnmarge pro verkauften Modell trotzdem vor dem japanischen Massenhersteller Toyota, der aktuell die Spitze an verkauften Modellen pro Jahr führt.\\
Der Hersteller entwickelt ausschließlich Sportwagen, Oberklassenmodelle und SUV's.
Die gängisten Serien des Herstellers sind Cayenne, Boxter und 911, von denen 911 mit Carriera die größte Unterserie besitzt.\\
\\
\textbf{Lamborghini}\\
Der Luxushersteller produziert hauptsächlich Sportwagen und Coupes. Neben den Serienmodellen wurden auch eine
Vielzahl von Einzelmodellen entworfen, die in Design und Ergonomität, dafür aber auch Preis glänzen.
\newpage
\textbf{Bentley}\\
Der Hoflieferat für die brittische Königsfamilie, der ursprünglich lediglich als Schwestermarke der Firma Rolls-Royce bekannt war, ist seit  ist ebenfalls für seine Modelle in der Ober- und Sportklasse bekannt.\\
\\
\textbf{Bugatti}\\
Die Luxusmarke Bukatti versucht mit zwei Serien, Veyron 16.4 und 16C Galibier, die Kraft von meist 12-Zyllindrigen Motoren in ein dennoch komfortables Reisefahrzeug zu integrieren. Die Preise der Modelle gehen in die Millionen.\\
\\
\textbf{Ducati}\\
Über den Tochterkonzern Audi kam die Mottorad Marke Ducati in das Leben der Volkswagen Group. Die Produktpalette erstreckt sich hier vom leichten Stadtbike bis zur schweren Rennmaschiene, mit denen sich die Marke im Motorsport einen großen Namen gemacht hat.\\
\\
\textbf{Scania}\\
Das Sortiment des schwedischen herstellers bietet Trucks für den simplen Transport bis hin zur Sonderfertigung von Einsatzfahrzeugen. Die Modelle des Herstellers umfassen auschließlich Fahrzeuge für den Personen und Gütertransport. Stadt- und Reisebusse für Kurz- und Langstrecken sind ebenfalls mit inbegriffen.\\
Das Unternehmen arbeitet bereits länger eng mit der Volkswagen Group eigenen LKW-Tochter MAN zusammen und in näherer Zukunft gibt es pläne für eine Integration der beiden Hersteller. Siehe Ziele und Zukunftspläne.\\
\\
\textbf{MAN}\\
MAN ist die eigene LKW-Tochter der Volkswagen Group und somit der zweite Hersteller für Transport- und Reisefahrzeuge im Bund. Das Angebot umfasst Transporter für Güter, Sattelschlepper, Reise und Stadtbusse.\\
\\
\textbf{Volkswagen Financial Services}\\
Die Dienstleistungen der Volkswagen Financial Services beschäftigen sich hauptsächlich mit der Finanzierung, Leasing und Versicherung von Fahrzeugen eines Volkswagen Kunden. Zusätzlich kommen noch Dienstleistungen wie Mobilitätsgarantie und direktes Banking hinzu. \cite{vw-produkte} \cite{audiRinge} \cite{audiNeuwagen} \cite{vwPorscheUebernahme}

\newpage
\subsection{Standorte}
Der Konzern betreibt über 118 Fertigungsstätten in denen rund 600.000 Personen beschäftigt sind.
Der Kern des Unternehmnes sitzt in Deutschland, wobei Niedersachsen mit einem minimalen Überschuss die Sperrminorität und somit ein Vetorecht auf alle wichtigen Entscheidungen besitzt.
Der überwiegende Teil der Produktionsstätten ist nach Afrika und Asien ausgelagert, der rest wird zum Teil in Amerika und Europa produziert.\\
Die folgenden Ländern gibt die Volkswagen Group an zu produzieren: \cite{produktionsstandorte}
\begin{table}[h]
	\begin{tabular}{|l|l|l|l|l|}
		\hline
		Argentinien          & Bosnien und Herzegovina & Brasilien & China    & Deutschland          \\ \hline
		Indien               & Mexiko                  & Polen     & Portugal & Russische Föderation \\ \hline
		Slowakische Republik & Spanien                 & Südafrika & USA      &                      \\ \hline
	\end{tabular}
\end{table}
\\

\subsection{Unternehmensstruktur}
"Die Volkswagen AG stellt die Muttergesellschaft des eigentlichen Volkswagen Konzerns da und entwickelt insbesondere Pkw und Nutzfahrzeuge für den Vertrieb, sowie Fahrzeuge und deren Komponenten für den Konzern. Der Vorstand der Volkswagen AG leitet das Unternehmen in eigener Verantwortung. Der Aufsichtsrat bestellt, überwacht und berät den Vorstand und ist in Entscheidungen, die von grundlegender Bedeutung für das Unternehmen sind, unmittelbar eingebunden." \cite{struktur}

\subsubsection{Vorstand und Konzernleitung}
Das Gremium Konzernleitung trägt dafür Sorge, dass die Konzerninteressen bei Entscheidungen der Marken und Gesellschaften des Konzerns beachtet werden. Es besteht aus den Mitgliedern des Vorstands und ausgewählten Top-Managern mit Konzernsteuerungsfunktionen.\cite{struktur}
\newpage
\subsubsection{Vorsitzende des Vorstands der Volkswagen AG}
Obwohl das Unternehmen immer mehr Internationalität anstrebt und besonders in Amerika einen immer besseren Eindruck vermitteln möchte, kommt ein sehr großer Teil des neun Mann großen Vorstands aus Deutschland. Das liegt unter anderem daran das Volkswagen auch heute noch einen sehr großen Teil seiner Geschäfte in deutschsprachigen beziehungsweise europäischen Ländern abwickelt.\cite{vorstand} \\ \\
\textbf{Vorstands Vorsitzender}
\begin{figure}[!h]
	\centering
	\begin{minipage}[h]{0.20\textwidth}
		\centering
		\includegraphics[width=1.0\textwidth]{images/MartinWinterkorn.jpg}
		\label{fig:vorstandvw0}
	\end{minipage}
		\begin{minipage}[h]{0.10\textwidth}
		\hspace{1cm} 
	\end{minipage}
	\begin{minipage}[h]{0.65\textwidth}
		Prof. Dr. Dr. h. c. mult. Martin Winterkorn ist Vorsitzender des Vorstands der Volkswagen AG. Sein Geschäftsbereich ist 'Konzern Forschung und Entwicklung'. Zusätzlich dazu ist er seit 1. Januar 2007 Vorsitzender des Aufsichtsrats der AUDI AG und seit 25. November 2009 Vorsitzender des Vorstands Porsche Automobil Holding. 		(Bild: \cite{mwpic} )

	\end{minipage}
\end{figure}\FloatBarrier\noindent

\textbf{Mitglieder des Vorstands}
\begin{figure}[!h]
	\centering
	\begin{minipage}[h]{0.65\textwidth}
		Dr. rer. pol. h. c. Francisco Javier Garcia Sanz ist seit dem 1. Juli 2001 Vorstandsmitglied der Volkswagen AG. Sein Geschäftsbereich ist 'Beschaffung'. Zusätzlich ist er schon seit dem 30. November 1996 Vorsitzender der Marke Volkswagen PKW bei der er ebenfalls dem Geschäftsbereich 'Beschaffung' zugeteilt ist. Seit Juni 2007 ist Francisco Javier Garcia Sanz Vorsitzender des Verwaltungsrats der SEAT, S.A. (Barcelona). (Bild: \cite{fspic} )
	\end{minipage}
		\begin{minipage}[h]{0.10\textwidth}
		\hspace{1cm}
	\end{minipage}
	\begin{minipage}[h]{0.20\textwidth}
		\centering
		\includegraphics[width=1.0\textwidth]{images/FranciscoSanz.jpg}
		\label{fig:vorstandvw1}
	\end{minipage}
\end{figure}

\begin{figure}[!h]
	\centering
	\begin{minipage}[h]{0.20\textwidth}
		\centering
		\includegraphics[width=1.0\textwidth]{images/JochemHeizmann.jpg}
		\label{fig:vorstandvw2}
	\end{minipage}
	\begin{minipage}[h]{0.10\textwidth}
		\hspace{1cm} 
	\end{minipage}
	\begin{minipage}[h]{0.65\textwidth}
		Der Aufsichtsrat der Volkswagen AG beruf Prof. Dr. rer. pol. Dr.-Ing. E. h. Jochem Heizmann am 1. September 2012 zu einem Mitglied des Vorstands für den Geschäftsbereich 'China'. Er nimmt diese Position zusätzlich zu seiner Funktion als President und CEO der Volkswagen Group China wahr. Vom 1. Oktober 2010 bis zum 31. August 2012 verantwortete Prof. Dr. Heizmann weiters als Mitglied des Vorstands der Volkswagen AG den Geschäftsbereich 'Konzern Nutzfahrzeuge'. (Bild: \cite{jhpic} )
	\end{minipage}
\end{figure}

\begin{figure}[!h]
	\centering
	\begin{minipage}[h]{0.65\textwidth}
		Von 2004 bis 2008 war Christian Klingler Mitglied der Geschäftsführung der Porsche Holding Österreich.
		Im August 2008 wurde Christian Klingler als Generalbevollmächtigter der Volkswagen AG in den Vorstand der Marke Volkswagen berufen und war verantwortlich für die Bereiche Vertrieb und Marketing sowie After Sales. Seit dem 1. Januar 2010 ist Klingler zusätzlich Mitglied des Vorstands der Volkswagen AG und zuständig für den Geschäftsbereich 'Vertrieb und Marketing'  (Bild: \cite{ckpic} )
	\end{minipage}
	\begin{minipage}[h]{0.10\textwidth}
		\hspace{1cm} 
	\end{minipage}
	\begin{minipage}[h]{0.20\textwidth}
		\centering
		\includegraphics[width=1.0\textwidth]{images/ChristianKlingler.jpg}
		\label{fig:vorstandvw3}
	\end{minipage}
\end{figure}

\begin{figure}[!h]
	\centering
	\begin{minipage}[h]{0.20\textwidth}
		\centering
		\includegraphics[width=1.0\textwidth]{images/MathiasMueller.jpg}
		\label{fig:vorstandvw4}
	\end{minipage}
	\begin{minipage}[h]{0.10\textwidth}
		\hspace{1cm} 
	\end{minipage}
	\begin{minipage}[h]{0.65\textwidth}
		Von 2007 an war Matthias Müller Leiter des Produktmanagements des Volkswagen Konzerns und der Marke VW sowie Generalbevollmächtigter des Volkswagen Konzerns. Matthias Müller ist seit 2010 Vorstandsvorsitzender der Dr. Ing. h.c. F. Porsche AG sowie Vorstandsmitglied der Porsche Automobil Holding SE.
		In seiner Funktion als Vorstandsvorsitzender der Dr. Ing. h.c. F. Porsche AG wurde Matthias Müller zum 1. März 2015 in den Vorstand der Volkswagen Aktiengesellschaft berufen. (Bild: \cite{mmpic} )
	\end{minipage}
\end{figure}

\begin{figure}[!h]
	\centering
	\begin{minipage}[h]{0.65\textwidth}
		Seit Juli 2002 ist Dr. Horst Neumann Mitglied des Vorstands der AUDI AG und damit verantwortlich für den Geschäftsbereich Personal- und Sozialwesen. Zum 1. Dezember 2005 bestellte ihn der Aufsichtsrat der Volkswagen AG zum Mitglied des Vorstands der Volkswagen AG  der Marke Volkswagen für den Bereich "Personal, Arbeitsdirektor". Seit Januar 2007 ist er verantwortlich für Personal und Organisation. (Bild: \cite{hmpic} )
	\end{minipage}
	\begin{minipage}[h]{0.10\textwidth}
		\hspace{1cm} 
	\end{minipage}
	\begin{minipage}[h]{0.20\textwidth}
		\centering
		\includegraphics[width=1.0\textwidth]{images/HorstNeumann.jpg}
		\label{fig:vorstandvw5}
	\end{minipage}
\end{figure}

\begin{figure}[!h]
	\centering
	\begin{minipage}[h]{0.20\textwidth}
		\centering
		\includegraphics[width=1.0\textwidth]{images/AndreasRenschler.jpg}
		\label{fig:vorstandvw6}
	\end{minipage}
	\begin{minipage}[h]{0.10\textwidth}
		\hspace{1cm} 
	\end{minipage}
	\begin{minipage}[h]{0.65\textwidth}
		Früher wurde Andreas Renschler bei Mitsubishi Motors in Japan und der Daimler AG im Bereich Daimler Trucks eingesetzt. Seit 1. Februar 2015 ist er Mitglied des Vorstands der Volkswagen AG im Geschäftsbereich 'Nutzfahrzeuge'. (Bild: \cite{arpic} )
	\end{minipage}
\end{figure}

\begin{figure}[!h]
	\centering
	\begin{minipage}[h]{0.65\textwidth}
		"Der Aufsichtsrat der Volkswagen AG übertrug Dipl. Wirtsch.-Ing. Hans Dieter Pötsch mit Wirkung zum 5. September 2003 die Verantwortung für den Geschäftsbereich ‘Finanzen und Controlling‘ auf Konzernebene im Vorstand. Bereits zum 1. Januar 2003 hatte der Aufsichtsrat Pötsch zum ordentlichen Mitglied des Vorstands, zunächst ohne Geschäftsbereich, berufen. Seit dem 25. November 2009 ist Hans Dieter Pötsch zusätzlich zu seinen bisherigen Funktionen im Vorstand der Porsche Automobil Holding SE, Stuttgart, vertreten. Er nimmt dort ebenfalls die Position des Finanzvorstands ein."\cite{poetschbio}  (Bild: \cite{dppic} )
	\end{minipage}
	\begin{minipage}[h]{0.10\textwidth}
		\hspace{1cm} 
	\end{minipage}
	\begin{minipage}[h]{0.20\textwidth}
		\centering
		\includegraphics[width=1.0\textwidth]{images/HansPoetsch.jpg}
		\label{fig:vorstandvw7}
	\end{minipage}
\end{figure}

\begin{figure}[!h]
	\centering
	\begin{minipage}[h]{0.20\textwidth}
		\centering
		\includegraphics[width=1.0\textwidth]{images/RupertStadler.jpg}
		\label{fig:vorstandvw8}
	\end{minipage}
	\begin{minipage}[h]{0.10\textwidth}
		\hspace{1cm} 
	\end{minipage}
	\begin{minipage}[h]{0.65\textwidth}
		Prof. Rupert Stadler ist seit dem 1. Januar 2003 Mitglied des Vorstandes der AUDI AG. Am 1. April 2003 übernahm er die Verantwortung für den Geschäftsbereich Finanz und Organisation. Seit 1. Januar 2007 ist Rupert Stadler Vorsitzender des Vorstands der AUDI AG. Den Geschäftsbereich Finanz und Organisation führte er in Personalunion bis 31. August 2007 weiter. Rupert Stadler wurde in seiner Funktion als Vorsitzender des Vorstands der AUDI AG zum 1. Januar 2010 in den Vorstand der Volkswagen Aktiengesellschaft berufen. (Bild: \cite{rspic} )
	\end{minipage}
\end{figure}
\cite{vorstand}

\subsubsection{Marken und Hierarchie}
Jede Marke des Volkswagen Konzerns wird von einem Markenvorstand geleitet. Dabei sind die vom Vorstand der Volkswagen AG beziehungsweise von der Konzernleitung festgelegten Konzernziele und -vorgaben und die jeweiligen rechtlichen Rahmenbedingungen zu beachten. Angelegenheiten von konzernweiter Bedeutung werden der Konzernleitung vorgelegt.
Die einzelnen Gesellschaften des Volkswagen Konzerns werden jeweils unter Verantwortung einer eigenen Geschäftsleitung geführt. Dabei berücksichtigen die jeweiligen Geschäftsleitungen neben den Interessen der Gesellschaft auch Konzern- und Markeninteressen.\cite{structure1}
Die Hierarchie sieht dabei wie folgt aus:

\begin{table}[h]
	\begin{tabular}{|c|c|c|c|c|c|c|c|c|}
		\hline
		\multicolumn{9}{|c|}{Volkswagen AG}                                                                                                                                                     \\ \hline
		\multicolumn{3}{|c|}{Volkswagen} & \multicolumn{2}{c|}{Audi} & \multicolumn{2}{c|}{\begin{tabular}[c]{@{}c@{}}Volkswagen\\ Nutzfahrzeuge\end{tabular}} & \multicolumn{2}{c|}{Weitere} \\ \hline
		Skoda    & Bugatti   & Bentley   & Seat     & Lamborghini    & Scania                                       & MAN                                      & Suzuki       & Porsche       \\ \hline
	\end{tabular}
\end{table}

Die Volkswagen AG ist grundsätzlich schon immer durch die starke Eigenständigkeit der verschiedenen Marken geprägt. Die Marken sind beispielsweise für Produktdesign, Vertrieb, Marketing und vieles mehr weitgehend eigenständig. Sogar die Produktionsstandorte und Werke sind meistens einer einzelnen Marke zugeteilt und werden von dieser organisiert.
\cite{domavw}
\newpage
\subsection{Logistik}
\begin{figure}[!h]
	\centering
	\includegraphics[width=0.7\textwidth]{images/logologistics.jpg}
	\caption{Logo der Volkswagen Logistics GmbH \& Co. OHG}
	\label{fig:vwlogisticspic}
	\cite{vwlogisticspic}
\end{figure}\FloatBarrier
Die Logistikprozesse der Volkswagen AG werden durch den Dienstleister Volkswagen Logistics GmbH \& Co. OHG gesteuert und verwaltet.
Volkswagen Logistics übernimmt hierbei nicht nur integrierte Logistik bei Volkswagen, sondern auch externe Aufträge für Dienstleister und Kunden.\cite{vwlogistics}

\subsubsection{Leitung und Sitz}
Volkswagen Logistics GmbH \& Co. OHG hat seinen Sitz in Wolfsburg und wird von Thomas Zernechel und Dinah J. Kamiske geleitet.
Das Unternehmen agiert mit seinen 600 Mitarbeitern international auf über 130 Märkten.\cite{vwlogistics}

\subsubsection{Zuständigkeit}
Volkswagen Logistics betreut dabei die gesamte Supply-Chain, angefangen beim Lieferanten, über die Produktions- und Distributionsstufen, bis hin zur Auslieferung beim Kunden.\\
Volkswagen Logistics ist außerdem für die Lagerung, den Transport und die Verpackung von Produkten zuständig. Zusätzlich übernimmt Volkswagen Logistics auch die Entsorgungslogistik, die sowohl bei der normalen Produktion, als auch bei der Entwicklung von Prototypen eine sehr wichtige Rolle spielt.\cite{vwlogistics}

\subsubsection{Besonderheiten}
43 Milliarden Teile werden jährlich allein innerhalb den Volkswagen Konzerns transportiert. Um dies zu ermöglichen ist daher eine starke und flexibel agierende Logistik unverzichtbar.
Innovation und Ideen sind bei Volkswagen Logistics schon immer sehr wichtig. Volkswagen Logistics ist überzeugt das nur mithilfe modernster Technologien und Optimierungen ein Wettbewerbsvorteil erreicht werden kann.

Im Rahmen einer Fachtagung, des so genannten "Innovationstag Logistik", stellte Volkswagen 2013 einige ihrer Besonderheiten vor. Bemerkenswerte Technik mit denen Volkswagen arbeitet sind unter anderem lasergesteuerte, fahrerlose Transportsysteme zur automatisierten Materialbereitstellung, ergonomische Handschubwagen mit Elektroantrieb und besondere Roboter, die eigens für bestimmte Logistikprobleme entwickelt wurden.\cite{wolfsburg}
\subsection{Plattform Strategie}
Volkswagen verwendet schon seit sehr langer Zeit sogenannte modulare Baukastenstrategien um die Kosten der Fahrzeugentwicklung zu senken. Das neueste, 2012 eingeführte, Mitglied der Baukastenfamilie ist der Modulare Querbaukästen (kurz MQB). Er ist die Weiterentwicklung der Plattform- und Modulstrategie die Volkswagen gegen Mitte der 90er Jahre entwickelt hat. Der MQB wird als Basis für Fahrzeuge, deren Motor quer zur Fahrtrichtung gebaut ist, verwendet. Der neue Audi A3, der neue Golf, der SEAT Leon und der SEAT Octavia basieren unter anderem auf diesem modularem Baukasten. Weiters gibt es noch viele andere modulare Baukästen, zum Beispiel der sehr ähnliche Modulare Längsbaukasten (MLB).\cite{mqb}
\FloatBarrier
\begin{figure}[!h]
	\centering
	\includegraphics[width=0.7\textwidth]{images/MQB}
	\caption{Verschiedene Baukästen für verschiedene Klassen \cite{baukastengraph}}
\end{figure}\FloatBarrier\noindent

Hinter diesen modularen Baukästen steckt die Idee das Autos einander ähneln. Auch wenn Autos sich optisch oder auch funktional unterscheiden, bleiben viele Teile sehr ähnlich. Deshalb wurden sogenannte Plattform Strategien (von Volkswagen Baukastenstrategien genannt) entwickelt. Diese definieren Grundgerüste die zur Entwicklung verschiedenster Autos verwendet werden. \\ "Mit dem MQB ist eine hochflexible Fahrzeugarchitektur entstanden, bei der konzeptbestimmende Abmessungen wie Radstand, Spurbreite, Rädergröße und Sitzposition konzernweit abgestimmt sind und variabel zum Einsatz kommen. Andere Abmessungen, zum Beispiel der Abstand der Pedale zur vorderen Radmitte, sind immer gleich und gewährleisten eine einheitliche Systematik des Vorderwagens."\cite{mqb}
\FloatBarrier
\begin{figure}[!h]
	\centering
	\includegraphics[width=0.7\textwidth]{images/MQB2}
	\caption{Grundarchitektur eines MQB-Fahrzeugs \cite{mqbdefault}}
\end{figure}\FloatBarrier
\newpage\newpage
%TODO LAYOUT
\subsection{Markt}
%Marktumfeld, -situation, -wert, Trends
Die Volkswagen Group, also alle Untermarken von VW gemeinsam, produziert global am meisten PKWs, jedoch führen Toyota sowie General Motors noch vor Volkswagen den globalen, totalen Verkauf an Fahrzeugen aller Klassen. \\
In 2013 wurden von Volkswagen etwa 9.3 Millionen Autos produziert. \\
Toyota kommt aus Japan und dominiert sowohl den japanischen als auch den amerikanischen Markt. GM kommt aus Amerika und produziert eher Pick-Ups - daher dominieren sie hier traditionell den amerikanischen Markt, obwohl Volkswagen mit den Amarok Modellen eigentlich das erforderliche Produkt hätte. 
\\\\\textbf{Weltweit}\\
Wie in Abbildung \ref{fig:marktwelt} ersichtlich ist, führt Volkswagen den Marktanteil bei PKWs weltweit an. 

%\begin{figure}[!h]
%\centering
%\begin{minipage}[h]{0.40\textwidth}
%\centering
%\includegraphics[width=1.0\textwidth]{images/maww}
%\caption{Marktanteil Weltweit (PKWs)}
%\label{fig:marktwelt}
%\end{minipage}
%\begin{minipage}[h]{0.08\textwidth}
%\centering
% \hbox{}
%\end{minipage}
%\begin{minipage}[h]{0.50\textwidth}
%\centering
%\includegraphics[width=1.0\textwidth]{images/maie}
%\caption{Marktanteil in Europa (PKWs)}
%\label{fig:markteuropa}
%\end{minipage}
%\end{figure}
%\FloatBarrier




\begin{figure}[!h]
\centering
\includegraphics[width=0.5\textwidth]{images/maww}
\caption{Der Marktanteil Weltweit (PKWs)}
\label{fig:marktwelt}
\end{figure}\FloatBarrier
\noindent
\textbf{Europa}\\
Wie in Abbildung \ref{fig:markteuropa} ersichtlich ist, führt Volkswagen den Marktanteil in Europa bei weitem an. Diese Tendenz ist steigend.
\begin{figure}[!h]
\centering
\includegraphics[width=0.7\textwidth]{images/maie}
\caption{Der Marktanteil in Europa (PKWs)}
\label{fig:markteuropa}
\end{figure}\FloatBarrier
\noindent
\textbf{Asien}\\
"Seit Beginn seiner Reform- und Öffnungspolitik vor über 30 Jahren ist China zu einem der wichtigsten Automobilmärkte der Welt aufgestiegen und bildet inzwischen den größten Absatzmarkt des Volkswagen Konzerns. Mit einem Anteil von 20,8 \% am Pkw-Markt und 3,27 Millionen verkauften Fahrzeugen im Geschäftsjahr 2013 ist der Volkswagen Konzern Marktführer in China."\cite{vwwebsitechina}\\
Volkswagen war einer der ersten Autohersteller die China nicht nur als Absatzmarkt sondern auch als Produktionsstandort betrachtet hat und daher stark mit der kommunistischen Führung in China kooperiert hatte. Dadurch konnte Volkswagen sich bis heute in Asien, insbesondere jedoch in China sehr gut positionieren: 
"Die kommunistische Führung in China lässt ausländische Investoren aus Schlüsselbranchen wie der Autoindustrie nur mit inländischen Partnern agieren - in Gemeinschaftsunternehmen (Joint Ventures). Damit soll verhindert werden, dass die ausländischen Marken den Markt alleine dominieren." \cite{vwamstrategiechina2}\\
Durch das dort ansässige Unternehmen Shanghai-Volkswagen Automotive Company (SVW) sowie des Unternehmens FAW-Volkswagen Automotive Company Ltd. (FAW-VW) konnten daher viele Produktionsstandorte in China aufgebaut werden.
Asien, insbesondere China, zeichnet sich durch eine stetig steigende Nachfrage aus. So wurden inzwischen etwa 17 Gesellschaften (Komponenten-, Finanz- und Vertriebsgesellschaften) in Asien gegründet. \\
Ende der 90er Jahre wurde auch in Asien verstärkt mit der Diversifikation der Produktpalette begonnen. Das  Produktportfolio in Asien umfasst heute alle Segmente vom Kleinwagen bis zum Luxussportwagen.
\\
Mit seinem umfangreichen Produktportfolio, neuesten Technologien sowie den Planungen zu Hybrid- und Elektrofahrzeugen ist der Volkswagen Konzern bestens gerüstet, um zukünftigen Herausforderungen – zum Beispiel anspruchsvollen Emissionsgrenzen oder Zulassungsbeschränkungen in Megacitys – zu begegnen und auch langfristig eine Schlüsselrolle auf dem Automobilmarkt in Asien innezuhaben. \\
In China herrscht des weiteren ein sehr ausgeprägtes Marken bewusstsein und die Volkswagen Group gilt dort als guter europäischer Fahrzeugbauer.  Eine Strategie von Volkswagen um in Asien noch besser positioniert zu sein, ist eher auf die individuellen Bedürfnisse der Asiaten (Chinesen sind zum Beispiel in der Regel kleiner und leichter) einzugehen. \cite{vwamstrategiechina}
\begin{figure}[!h]
\centering
\includegraphics[width=0.45\textwidth]{images/mach}
\caption{Der Marktanteil in China (PKWs)}
\label{fig:marktchina}
\end{figure}\FloatBarrier
\noindent
\textbf{Japan} \\
Volkswagen konnte in Japan kaum Fuß fassen, unter anderem auch da Japan einige eigene Marken hervorgebracht hat.  Die Volkswagen Group kommt in Japan nur unter der Kategorie 'Andere' vor. Volkswagen selbst sieht jedoch das japanische Marktsegment als nicht erstrebenswert zu erklimmen, da es sehr schwierig ist die lokalen Anbieter dort zu vertreiben.
\\\\
\textbf{Nordamerika}\\
Volkswagen konnte sich in Amerika noch nicht durchsetzten. Derzeit hat Volkswagen nur 4\% des Marktanteiles (Verglichen mit 25\% in Europa) inne, wie man in Abbildung \ref{fig:marktnordamerika} sehen kann. Dies zu steigern ist ein Ziel der kommenden paar Jahre. Die Autos haben zwar als deutsche Qualitätsware einen guten Ruf, jedoch fehlt in Amerika auch die nötige Infrastruktur, also Händler und Werkstätten. \\ Des weiteren werden in Amerika mehr SUVs und Pick-Ups gekauft, ein Segment in welchem Volkswagen nicht unbedingt stark vertreten ist.\\
Trotzdem konnte Volkswagen 2012 die besten Verkaufszahlen seit dem Käfer erwirtschaften und laut eigenen Angaben sollen ab 2018 mindestens 800.000 Autos pro Jahr verkauft werden. In den ersten fünf Monaten von 2014 jedoch ist der Verkauf wieder um 15\% gefallen, zu Gunsten von GM. Jedoch erwartet Volkswagen, dass mit der Einführung des neuen Golfes ihr Marktanteil steigen wird. Porsche und Audi verkaufen sich sehr gut in Nordamerika. \cite{vwamstrategiechina} \\
Mit nur circa 400.000 verkauften Neuwagen in 2013 - um 7\% weniger als im Vorjahr - ist es nicht klar, ob das Ziel bis 2018 in Amerika noch erreicht werden kann.\cite{ec1}\\
Amerika ist daher der - neben Japan - am schwierigsten zu erschliessende Markt.
\begin{figure}[!h]
\centering
\includegraphics[width=0.7\textwidth]{images/maam}
\caption{Der Markt in den USA als Beispiel für Nordamerika (PKWs)}
\label{fig:marktnordamerika}
\end{figure}\FloatBarrier
\noindent
\\\\
\textbf{Südamerika}\\
In Südamerika geht es Volkswagen relativ gut. Fiat führt zwar den Markt an, aber dicht gefolgt von Volkswagen. \\
Jedoch hatten sie dort eine Umsatzeinbusse in den letzten Jahren.







\subsection{Ziele und Zukunftsaspekte}
Laut der Volkswagen Website steht bis 2018 die Positionierung des Konzerns als weltweit führend sowohl ökonomisch und ökologisch an erster Stelle.\\ Das Toyota 2015 allerdings mit Einbußen im Heimatland Japan rechnet könnte es VW 2015 schaffen erstmals der größte Autohersteller der Welt zu werden - dieses Ziel wäre daher schon 2015 anstatt erst 2018 verwirklicht worden.\\
Zur Erreichung dieser Ziele wurden vier offizielle Subziele (\cite{vwwebsitestrat}) definiert:
\begin{enumerate}
\item Steigerung der Kundenzufriedenheit und Qualität. Erreichung einer höheren Kundenzufriedenheit für nachhaltigen Erfolg.
\item Der Absatz soll von 9.259 Mio Fahrzeuge auf mehr als 10 Mio. Fahrzeuge gesteigert werden.  Vor allem auf die expandierenden Märkte in Asien sowie der noch schwach genützte Markt von Nordamerika soll ein stärkeres Augenmerk gelegt werden.
\begin{figure}[!h]
\centering
\includegraphics[width=0.4\textwidth]{images/strategie}
\caption{ Wachstumsstrategie \cite{vwamstrategie3}}
\end{figure}\FloatBarrier
\noindent
\item Die Umsatzrendite vor Steuern soll nachhaltig mindestens 8\% betragen, damit die finanzielle Stabilität und Handlungsfähigkeit des Konzerns auch in schwierigen Marktphasen sichergestellt ist. 
\item Volkswagen will sich auch als Arbeitgeber besser profilieren, um so höher qualifizierte Mitarbeiter gewinnen zu können.  \\\\
Des weiteren sind folgende Ziele immer wieder zu erkennen: \\
\item Volkswagen will auf dem amerikanischen Markt stärker präsent sein. Um genau dies zu schaffen, muss Volkswagen den Giganten Toyota vertreiben. \cite{ec1}
\item Volkswagen will in wachsenden Märkten vor allem mit dem neuen e-Golf und dem e-Up! Position beziehen. Elektro Fahrzeuge wurden Anfang 2000 noch kategorisch von der Firmenspitze abgelehnt. \cite{ec3} 
\item Es sollen umweltverträglichere und vor allem spritsparende Autos entwickelt und produziert werden, um so mit dem derzeitigem globalen Trend von Nachhaltigkeit mithalten zu können. Durch ein Baukastensystem sowie einen Leichtbau des Fahrzeuges sollen neue ökologische Maßstäbe gesetzt werden.
\item Natürlich soll die führende Position von Volkswagen in Europa beibehalten werden und finanzielle Rücklagen ausreichend gebildet werden.
\item Das Wettrennen gegen GM und Toyota soll weniger stark angegangen werden, so dass Volkswagen will zukünftig versuchen seine Gewinnspanne pro Auto zu steigern. Sie wollen mehr auf das Baukastensystem setzten. So sollen die Stückkosten um 20\% gesenkt werden und die Bauzeit pro Auto um 30 \% verkürzt werden. \cite{vwamstrategie}
\item Die Volkswagengroup besitzt mit der bereits einen überwiegenden Teil an Stimmrechten und Kapital der schwedischen Marke Scanias und plant eine Integration mit der eigenen LKW-Tochter MAN. Dieser Entschluss wird derzeit allerdings von Experten der International Strategy \& Investment London sehr kritisch betrachtet, da die Komplettübernahme des Herstellers der Volkswagen Group 8.6 Milliarden Euro wert ist. Ein betrag der in keinem Verhältnis zu den dafür errechneten Profiten von 1.8 Milliarden Euro in den nächsten Jahren steht. Die Volkswagen Group rechtfertigt diese Transaktion mit dem Aufheben von Barrieren die das Optimum aus der Zusammenarbeit von Scania und MAN herausholen soll.
Im laufenden Jahr 2015 sollen 40 bis 50 Prozent der geplanten Übernahme aus dem Cashflow finanziert werden. \cite{ScaniaMAN}
\end{enumerate}
\newpage
\section{Auswahl eines ERP Systemes}
\subsection{Auswahlkriterien}
Wir haben die Auswahl mittels Trovarit IT-Matchmaker gemacht. Die folgenden Grafiken zeigen die von uns ausgewählten Anforderungen: \\
Kritische Anforderungen werden rot dargestellt, geforderte Anforderungen gelb und optionale Anforderungen werden grün dargestellt.
\begin{figure}[!h]
\centering
\includegraphics[width=0.7\textwidth]{images/tr1}
\end{figure}\FloatBarrier
\noindent
\begin{figure}[!h]
\centering
\includegraphics[width=0.7\textwidth]{images/tr2}
\end{figure}\FloatBarrier
\noindent
\begin{figure}[!h]
\centering
\includegraphics[width=0.7\textwidth]{images/tr3}
\end{figure}\FloatBarrier
\noindent
\begin{figure}[!h]
\centering
\includegraphics[width=0.7\textwidth]{images/tr4}
\end{figure}\FloatBarrier
\noindent
\begin{figure}[!h]
\centering
\includegraphics[width=0.7\textwidth]{images/tr5}
\end{figure}\FloatBarrier
\noindent
\begin{figure}[!h]
\centering
\includegraphics[width=0.7\textwidth]{images/tr6}
\end{figure}\FloatBarrier
\noindent
\begin{figure}[!h]
\centering
\includegraphics[width=0.7\textwidth]{images/tr7}
\end{figure}\FloatBarrier
\noindent
\begin{figure}[!h]
\centering
\includegraphics[width=0.7\textwidth]{images/tr8}
\end{figure}\FloatBarrier
\noindent
\begin{figure}[!h]
\centering
\includegraphics[width=0.7\textwidth]{images/tr9}
\end{figure}\FloatBarrier
\noindent
\begin{figure}[!h]
\centering
\includegraphics[width=0.7\textwidth]{images/tr10}
\end{figure}\FloatBarrier
\noindent
Der Vertrieb sowie die gesamte Auftragsabwicklung und das CRM wird bei Volkswagen von Händlern übernommen und muss daher nur Schemenhaft in dem ERP System abgebildet werden.
\begin{figure}[!h]
\centering
\includegraphics[width=0.7\textwidth]{images/tr11}
\end{figure}\FloatBarrier
\noindent
\begin{figure}[!h]
\centering
\includegraphics[width=0.7\textwidth]{images/tr12}
\end{figure}\FloatBarrier
\noindent
\begin{figure}[!h]
\centering
\includegraphics[width=0.7\textwidth]{images/tr13}
\end{figure}\FloatBarrier
\noindent
\begin{figure}[!h]
\centering
\includegraphics[width=0.7\textwidth]{images/tr14}
\end{figure}\FloatBarrier
\noindent
\begin{figure}[!h]
\centering
\includegraphics[width=0.7\textwidth]{images/tr15}
\end{figure}\FloatBarrier
\noindent
\begin{figure}[!h]
\centering
\includegraphics[width=0.7\textwidth]{images/tr16}
\end{figure}\FloatBarrier
\noindent
\begin{figure}[!h]
\centering
\includegraphics[width=0.7\textwidth]{images/tr17}
\end{figure}\FloatBarrier
\noindent
\begin{figure}[!h]
\centering
\includegraphics[width=0.7\textwidth]{images/tr18}
\end{figure}\FloatBarrier
\noindent
\begin{figure}[!h]
\centering
\includegraphics[width=0.7\textwidth]{images/tr19}
\end{figure}\FloatBarrier
\noindent
\begin{figure}[!h]
\centering
\includegraphics[width=0.7\textwidth]{images/tr20}
\end{figure}\FloatBarrier
\noindent

\begin{figure}[!h]
\centering
\includegraphics[width=0.7\textwidth]{images/tr21}
\end{figure}\FloatBarrier
\noindent
\begin{figure}[!h]
\centering
\includegraphics[width=0.7\textwidth]{images/tr22}
\end{figure}\FloatBarrier
\noindent
\begin{figure}[!h]
\centering
\includegraphics[width=0.7\textwidth]{images/tr23}
\end{figure}\FloatBarrier
\noindent
\begin{figure}[!h]
\centering
\includegraphics[width=0.7\textwidth]{images/tr24}
\end{figure}\FloatBarrier
\noindent
\begin{figure}[!h]
\centering
\includegraphics[width=0.7\textwidth]{images/tr25}
\end{figure}\FloatBarrier
\noindent
\begin{figure}[!h]
\centering
\includegraphics[width=0.7\textwidth]{images/tr26}
\end{figure}\FloatBarrier
\noindent
\begin{figure}[!h]
\centering
\includegraphics[width=0.7\textwidth]{images/tr27}
\end{figure}\FloatBarrier
\noindent
\begin{figure}[!h]
\centering
\includegraphics[width=0.7\textwidth]{images/tr28}
\end{figure}\FloatBarrier
\noindent
\begin{figure}[!h]
\centering
\includegraphics[width=0.7\textwidth]{images/tr29}
\end{figure}\FloatBarrier
\noindent
\begin{figure}[!h]
\centering
\includegraphics[width=0.7\textwidth]{images/tr30}
\end{figure}\FloatBarrier
\noindent
\begin{figure}[!h]
\centering
\includegraphics[width=0.7\textwidth]{images/tr31}
\end{figure}\FloatBarrier
\noindent
\begin{figure}[!h]
\centering
\includegraphics[width=0.7\textwidth]{images/tr32}
\end{figure}\FloatBarrier
\noindent
\begin{figure}[!h]
\centering
\includegraphics[width=0.7\textwidth]{images/tr33}
\end{figure}\FloatBarrier
\noindent
\begin{figure}[!h]
\centering
\includegraphics[width=0.7\textwidth]{images/tr34}
\end{figure}\FloatBarrier
\noindent
\begin{figure}[!h]
\centering
\includegraphics[width=0.7\textwidth]{images/tr35}
\end{figure}\FloatBarrier
\noindent
\begin{figure}[!h]
\centering
\includegraphics[width=0.7\textwidth]{images/tr36}
\end{figure}\FloatBarrier
\noindent
\begin{figure}[!h]
\centering
\includegraphics[width=0.7\textwidth]{images/tr37}
\end{figure}\FloatBarrier
\noindent
\begin{figure}[!h]
\centering
\includegraphics[width=0.7\textwidth]{images/tr38}
\end{figure}\FloatBarrier
\noindent
\begin{figure}[!h]
\centering
\includegraphics[width=0.7\textwidth]{images/tr39}
\end{figure}\FloatBarrier
\noindent
\begin{figure}[!h]
\centering
\includegraphics[width=0.7\textwidth]{images/tr40}
\end{figure}\FloatBarrier
\noindent

\begin{figure}[!h]
\centering
\includegraphics[width=0.7\textwidth]{images/tr41}
\end{figure}\FloatBarrier
\noindent
\begin{figure}[!h]
\centering
\includegraphics[width=0.7\textwidth]{images/tr42}
\end{figure}\FloatBarrier
\noindent
\begin{figure}[!h]
\centering
\includegraphics[width=0.7\textwidth]{images/tr43}
\end{figure}\FloatBarrier
\noindent
\begin{figure}[!h]
\centering
\includegraphics[width=0.7\textwidth]{images/tr44}
\end{figure}\FloatBarrier
\noindent
\begin{figure}[!h]
\centering
\includegraphics[width=0.7\textwidth]{images/tr45}
\end{figure}\FloatBarrier
\noindent
\begin{figure}[!h]
\centering
\includegraphics[width=0.7\textwidth]{images/tr46}
\end{figure}\FloatBarrier
\noindent
\begin{figure}[!h]
\centering
\includegraphics[width=0.7\textwidth]{images/tr47}
\end{figure}\FloatBarrier
\noindent
\begin{figure}[!h]
\centering
\includegraphics[width=0.7\textwidth]{images/tr48}
\end{figure}\FloatBarrier
\noindent
\begin{figure}[!h]
\centering
\includegraphics[width=0.7\textwidth]{images/tr49}
\end{figure}\FloatBarrier
\noindent
\begin{figure}[!h]
\centering
\includegraphics[width=0.7\textwidth]{images/tr50}
\end{figure}\FloatBarrier
\noindent
\begin{figure}[!h]
\centering
\includegraphics[width=0.7\textwidth]{images/tr51}
\end{figure}\FloatBarrier
\noindent
\begin{figure}[!h]
\centering
\includegraphics[width=0.7\textwidth]{images/tr52}
\end{figure}\FloatBarrier
\noindent
Die Postleitzahl der Zentrale in Wolfsburg (D) ist 38440.
\begin{figure}[!h]
\centering
\includegraphics[width=0.7\textwidth]{images/tr53}
\end{figure}\FloatBarrier
\noindent
\begin{figure}[!h]
\centering
\includegraphics[width=0.7\textwidth]{images/tr54}
\end{figure}\FloatBarrier
\noindent
\begin{figure}[!h]
\centering
\includegraphics[width=0.7\textwidth]{images/tr55}
\end{figure}\FloatBarrier
\noindent
\subsection{Auswahl}
Zuerst konnte uns Trovarit leider keinen Kandidaten liefern, da alle Lösungen mindestens ein Knock-Out Kriterium erfüllt haben.
\begin{figure}[!h]
\centering
\includegraphics[width=0.8\textwidth]{images/matching1}
\caption{Kein Ergebniss}
\label{fig:matching1}
\end{figure}\FloatBarrier
\noindent
Ohne die Knock out Kriterien zu berücksichtigen, war das Ergebniss unserer Analysen wie folgt:

\begin{itemize}
\item Microsoft Dynamics AX 2012 \\
		82\% Erfüllung mit eigenem Produkt\\
		97\% Erfüllung mit Partnerprodukt\\
		wenn über KCS.net Holding AG implementiert\\
		Nachteile: Eignet sich nicht für Banken und Versicherungsbetriebe, Unterstützt keine Vertriebsstrategien, übernimmt kein Facility Management, die Lösung ist nicht Datenbankunabhängig und unterstützt kein Marketing über Facebook, sowie kein Hindi, Griechisch, Koreanisch und Slowenisch.
		
\begin{figure}[!h]
\centering
\includegraphics[width=0.7\textwidth]{images/matching2}
\caption{Die Spinnennetz von  Microsoft Dynamics AX }
\end{figure}\FloatBarrier
\noindent
\item SAP Business All-in-One\\
		93\% Erfüllung mit eigenem Produkt\\
		96\% Erfüllung mit Partnerprodukt\\		
		wenn über IBYKUS AG für Informationstechnologie implementiert\\
		Nachteile: Unterstützt keine Bilanzerstellung nach OR (CH), keine Social Business Collaboration, keine Standardnotationen, sowie keine Digitale Fabrikplanung. Der Anbieter an sich hat keine Standorte in Wien, USA und China. Des weiteren ist die Lösung nicht für Android-basierte Geräte verfügbar.
		

\begin{figure}[!h]
\centering
\includegraphics[width=0.7\textwidth]{images/matching3}
\caption{Die Spinnennetz von SAP Business All-in-One}

\end{figure}\FloatBarrier
\noindent
\item Microsoft Dynamics NAV\\
		93\% Erfüllung mit eigenem Produkt\\
		96\% Erfüllung mit Partnerprodukt\\				
		wenn über Microsoft Österreich implementiert\\
		Nachteile: Unterstützt nur suboptimal mehr als 249 Arbeitsplätze, hat nicht in allen Ländern Gutachten sowie ist es auch nicht Datenbank unabhängig. 
		
\FloatBarrier
\begin{figure}[!h]
\centering
\includegraphics[width=0.7\textwidth]{images/matching4}
\caption{Die Spinnennetz von Microsoft Dynamics NAV}
\end{figure}\FloatBarrier
\noindent


\end{itemize}
Wir würden uns am ehesten für Microsoft Dynamics NAV entscheiden, da hier am meisten mit dem eigentlichen Produkt abgedeckt worden ist und da die Nachteile eigentlich eher minimal sind.\\
Nach kurzer Rücksprache betrefflich unserer Auswahl wurde uns jedoch SAP ERP 6.0 empfohlen, daher haben wir auch den Outcome dieses Produktes, wenn auch nicht als erste Wahl von Trovarit betrachtet:
 \\ \\
\textbf{ SAP ERP 6.0}\\
		75\% Erfüllung mit eigenem Produkt\\
		96\% Erfüllung mit Partnerprodukt\\				
		wenn über Agilita AG implementiert\\
		Nachteile: Unterstützt nur kein social business, kein Ticketing System, ist es nicht Datenbank unabhängig, und unterstützt weder Windows 2000 noch MAC OS. Agilita AG hat natürlich keinen Standort in Wien, Wolfsburg, USA oder China - jedoch kann man das Produkt SAP ERP 6.0 auch mit anderen Implementierenden Anbietern verwenden.
		
\FloatBarrier
\begin{figure}[!h]
\centering
\includegraphics[width=0.7\textwidth]{images/SAPBussinesNEU}
\caption{Die Spinnennetz von SAP ERP 6.0}
\end{figure}\FloatBarrier

%\section*{Projekthandbuch}
%\section*{Pflichtenheft}
%\section*{Vorbereitung Kick Off Meeting}

\small
\newpage\newpage
\begin{thebibliography}{56}

 \bibitem{ec1} 
  \textbf{Beetling back to success}, Jun 24th 2014, P.E, The Economist\\
  \textit{http://www.economist.com/blogs/schumpeter/2014/06/volkswagen-america}
  \newline last used: 07.03.2015, 13:00
  
   
 \bibitem{ec2} 
  \textbf{VW conquers the world - Germany’s biggest carmaker is leaving rivals in the dust}\\, Jul 7th 2012 , The Economist\\
  \textit{  http://www.economist.com/node/21558269}
  \newline last used: 07.03.2015, 13:07
  
    
   
 \bibitem{ec3} 
  \textbf{Europe goes electric - The Frankfurt motor show
}\\Sep 12th 2013 , P.E., The Economist\\
  \textit{  http://www.economist.com/node/21558269}
  \newline last used: 07.03.2015, 13:09
  
   \bibitem{vsc50y} 
  \textbf{Volkswagen Share celebrates its 50th birthday}\\ Jun 4th 2011 , Volkswagen AG\\
  \textit{http://www.volkswagenag.com/content/vwcorp/info\_center/en/themes/2011/04/\\Volkswagen\_Share\_celebrates\_its\_50th\_birthday.html
}
  \newline last used: 07.03.2015, 13:16
  
   \bibitem{2008kurs} 
  \textbf{Volkswagen Share celebrates its 50th birthday}\\ Jun 4th 2011 , Volkswagen AG\\
  \textit{  http://www.boerse.de/boersenwissen/boersengeschichte/Kurskapriolen-der-VW-Aktie-2008-\%7C45}
  \newline last used: 07.03.2015, 13:24
  

   \bibitem{jbilanz2013vw} 
  \textbf{ABSCHLUSS VOLKSWAGEN AG }, 2013 \\
  \textit{ http://www.volkswagenag.com/content/vwcorp/info\_center/de/publications/2014/03/\\Financial\_Statements\_VWAG\_2013.bin.html/binarystorageitem/file/Abschluss+Volkswagen+AG+2013\_deutsch.pdf  
}
  \newline last used: 08.03.2015, 14:03
    
    
       \bibitem{aktionaersstruktur} 
  \textbf{Aktionärsstruktur Volkswagen AG}, Stand 31.12.2014 \\
  \textit{http://www.volkswagenag.com/content/vwcorp/content/\\de/investor\_relations/share/Shareholder\_Structure.html}
  \newline last used: 01.04.2015, 15:04
    
  \bibitem{structure1} 
  \textbf{Struktur und Geschäftstätigkeit}\\
  \textit{http://www.volkswagenag.com/content/gb2007/content/de/corporate\_governance/\\structure\_and\_business\_activities\_\_part\_of\_the\_management\_report\_.html}
  \newline last used: 01.04.2015, 16:47
  
  \bibitem{domavw} 
  \textbf{Aufbauorganisation der Volkswagen AG}, volkswagenag.com \\
  \textit{http://bauhaus.cs.uni-magdeburg.de:8080/miscms.nsf/FEA8C8150500AA14C1257449004F79A9\\/CE79949E8E27681DC12579C1006D9A27/\$FILE/Diplomarbeit\%20Oliver\%20Meier.pdf}
  \newline last used: 01.04.2015, 18:51  
  
       \bibitem{2008wtf} 
 \textbf{Short sellers make VW the world's priciest firm}, reuters.com. SARAH Marsh \\
  \textit{  http://www.reuters.com/article/2008/10/28/us-volkswagen-idUSTRE49R3I920081028}
  \newline last used: 01.04.2015, 15:39  
  
         \bibitem{aktienfotos} 
 \textbf{Vorzugs- und Stammaktien}, volkswagenag.com \\
  \textit{   http://www.volkswagenag.com/content/vwcorp/content/de/investor\_relations/share.html}
  \newline last used: 01.04.2015, 15:57
  
  \bibitem{marken} 
 \textbf{Web Ressource}, marketbusinessnews.com \\
  \textit{   http://marketbusinessnews.com/wp-content/uploads/2014/02/Volkswagen-Group-Brands.png}
  \newline last used: 01.04.2015, 16:33
  
  \bibitem{produktionsstandorte} 
 \textbf{Produktionsstandorte}, mvolkswagenag.com \\
  \textit{   http://www.volkswagenag.com/content/vwcorp/content/de/the\_group/production\_plants.html}
  \newline last used: 01.04.2015, 17:02
  
  \bibitem{struktur} 
 \textbf{Struktur und Geschäftstätigkeit}, volkswagenag.com \\
  \textit{	http://www.volkswagenag.com/content/gb2007/content/de/corporate\_governance\\/structure\_and\_business\_activities\_\_part\_of\_the\_management\_report\_.html}
  \newline last used: 01.04.2015, 15:57  

      \bibitem{yahoofinanzenvw} 
 \textbf{Volkswagen AG},Yahoo Finanzen \\
  \textit{  https://de.finance.yahoo.com/q/ks?s=VOW3.DE}
  \newline last used: 01.04.2015, 16:33  
  
        \bibitem{vwwebsitestrat} 
 \textbf{Volkswagen AG Strategie},volkswagenag.com \\
  \textit{http://www.volkswagenag.com/content/vwcorp/content/de/the\_group/strategy.html}
  \newline last used: 01.04.2015, 17:00  
    
      \bibitem{vwwebsitechina} 
 \textbf{Markt Spezial: China},volkswagenag.com \\
  \textit{http://www.volkswagenag.com/content/vwcorp/content/de/investor\\\_relations/Warum\_Volkswagen/Marke\_Focus.html
}
  \newline last used: 01.04.2015, 17:38
  \bibitem{geschdautos} 
  \textbf{Geschichte des Autos} \\
  \textit{
  	http://www.fundus.org/referat.asp?ID=11034
  }
  \newline last used: 27.03.2015, 14:23
  
  
  \bibitem{autowp} 
  \textbf{Entstehungsgeschichte von VW} \\
  \textit{
  	http://www.autowallpaper.de/Wallpaper/VW/Entstehungsgeschichte\_VW.htm
  }
  \newline last used: 27.03.2015, 14:26
  
  \bibitem{terror} 
  \textbf{Wolfgang Benz, Barbara Distel}, 2005-2009 \\
  \textit{
  	Der Ort des Terrors. Geschichte der nationalsozialistischen Konzentrationslager
  }
  
  
  \bibitem{ahwest} 
  \textbf{Volgswagen AG Geschichte} \\
  \textit{
  	http://www.autohaus-westend.de/volkswagen-ag-geschichte/
  }
  \newline last used: 27.03.2015, 13:41
  
  \bibitem{vwag} 
  \textbf{Volgswagen AG} \\
  \textit{
  	http://www.volkswagenag.com/
  }
  \newline last used: 27.03.2015, 9:13
  
  \bibitem{sud} 
  \textbf{Drastische Gewinneinbruch bei VW} \\
  \textit{
  	http://www.sueddeutsche.de/wirtschaft/geschaeftsjahr-drastischer-gewinneinbruch-bei-vw-1.814556
  }
  \newline last used: 27.03.2015, 12:11
  
  \bibitem{vwchronik} 
  \textbf{VW Chronik} \\
  \textit{
  	http://www.chronik.volkswagenag.com/
  }
  \newline last used: 29.03.2015, 12:32

\bibitem{mwpic} 
\textbf{Martin Winterkorn Bild} \\
\textit{
	http://www.blogcdn.com/de.autoblog.com/media/2011/01/martin-winterkorn-vw-volkswagen.jpg
}
\newline last used: 02.04.2015, 14:01

\bibitem{fspic} 
\textbf{Francisco Sanz Bild} \\
\textit{
	http://autogramm.volkswagen.de/07-08\_12/images/content/popups/23\_Sanz.jpg
}
\newline last used: 02.04.2015, 14:01

\bibitem{jhpic} 
\textbf{Jochem Heizmann Bild} \\
\textit{
	http://autogramm.volkswagen.de/08\_10/images/content/popups/07\_P\_heizmann.jpg
}
\newline last used: 02.04.2015, 14:02

\bibitem{ckpic} 
\textbf{Christian Klingler Bild} \\
\textit{
	http://www.produktion.de/wp-content/uploads/2014/04/christian\_klingler.jpg
}
\newline last used: 02.04.2015, 14:03

\bibitem{mmpic} 
\textbf{Matthias Mueller Bild} \\
\textit{
	http://fotos.autozeitung.de/938x704/images/bildergalerie/2015/02/porsche-matthias-mueller.jpg
}
\newline last used: 02.04.2015, 14:04

\bibitem{hnpic} 
\textbf{Horst Neumann Bild} \\
\textit{
	http://www.astroman.com.pl/img/magazyn/1376/o/Horst\_Neumann\_1.jpg
}
\newline last used: 02.04.2015, 14:04

\bibitem{arpic} 
\textbf{Andreas Renschler Bild} \\
\textit{
	http://transportnet.se/files/2014/02/AndreasRenschler.jpg
}
\newline last used: 02.04.2015, 14:04

\bibitem{hppic} 
\textbf{Hans Poetsch Bild} \\
\textit{
	http://www.automobil-produktion.de/uploads/2011/03/vw\_poetsch-229x300.jpg
}
\newline last used: 02.04.2015, 14:05

\bibitem{rspic} 
\textbf{Rupert Stadler Bild} \\
\textit{
	http://www.matthiashaslauer.com/corporate/stadler/haslauer\_1.jpg
}
\newline last used: 02.04.2015, 14:05

\bibitem{vwlogistics} 
\textbf{Volkswagen Logistics} \\
\textit{
	http://www.volkswagen-logistics.com/
}
\newline last used: 02.04.2015, 14:59

\bibitem{vwlogisticspic} 
\textbf{Volkswagen Logistics Bild} \\
\textit{
	http://gvz-e-wolfsburg.de/images/logologistics.jpg
}
\newline last used: 02.04.2015, 15:02

\bibitem{vorstand} 
\textbf{Volkswagen AG Vorstand} \\
\textit{
	http://www.volkswagenag.com/content/vwcorp/content/de/the\_group/senior\_management.html
}
\newline last used: 02.04.2015, 15:02

\bibitem{wolfsburg} 
\textbf{Wolfsburg Fachtagung} \\
\textit{
	http://autogramm.volkswagen.de/09\_13/wolfsburg/wolfsburg\_01.html
}
\newline last used: 02.04.2015, 17:20

\bibitem{poetschbio} 
\textbf{Poetsch Biographie} \\
\textit{
	http://www.volkswagenag.com/content/vwcorp/content/de/the\_group/senior\_management/poetsch.html
}
\newline last used: 02.04.2015, 17:20

\bibitem{vw-produkte} 
\textbf{Volkswagen Navigator} \\
\textit{
	http://navigator.volkswagenag.com/index.html
}
\newline last used: 02.04.2015, 17:20


\bibitem{vwamstrategie} 
\textbf{Neue Volkswagen-Strategie: Winterkorns Wende},Michael Kröger \\
\textit{http://www.spiegel.de/wirtschaft/unternehmen/vw-jahresbilanz-volkswagen-chef-winterkorn-mit-neuer-strategie-a-958439.html}
\newline last used: 08.04.2015, 18:59


\bibitem{vwamstrategiechina} 
\textbf{Volkswagen will mit länderspezifischen Strategien an die Weltspitze}, dw.de \\
\textit{http://www.dw.de/volkswagen-will-mit-l\%C3\%A4\\nderspezifischen-strategien-an-die-weltspitze/a-17356433}
\newline last used: 08.04.2015, 19:18



\bibitem{vwamstrategiechina2} 
\textbf{VW drückt in China mächtig aufs Tempo}, Wirtschafts Woche \\
\textit{http://www.wiwo.de/unternehmen/auto/werke-sollen-schnell-eroeffnen-vw-entspricht-chinas-go-west-strategie/8488212-2.html}
\newline last used: 08.04.2015, 19:23

\bibitem{vwamstrategie3} 
\textbf{Jahrespressekonferenz 2012}, Volkswagen AG \\
\textit{http://www.volkswagenag.com/content/vwcorp/info\_center/de/talks\_and\_presentations/2012/03/JPK\_IK\_2012\_Part\_III.bin.html/binarystorageitem/file/Teil\_III\_Charts\_Winterkorn.pdf}
\newline last used: 08.04.2015, 19:40

\bibitem{mqb} 
\textbf{Volkswagen MQB} \\
\textit{
	http://www.volkswagenag.com/content/vwcorp/content/de/investor\_relations/Warum\_Volkswagen/MQB.html
}
\newline last used: 08.04.2015, 17:50

\bibitem{baukastengraph} 
\textbf{Volkswagen Baukästen} \\
\textit{
	http://www.volkswagenag.com/content/vwcorp/content/de/investor\\ \_relations/Warum\_Volkswagen/MQB.img.html/contentparsys/textandimages/images/textbinary/image/MQB1.png
}
\newline last used: 08.04.2015, 17:50

\bibitem{mqbdefault} 
\textbf{Volkswagen MQB} \\
\textit{
	http://www.volkswagenag.com/content/vwcorp/content/de/investor\\ \_relations/Warum\_Volkswagen/MQB.img.html/contentparsys/textandimages/images/textbinary\_0/image/MQB2.png
}
\newline last used: 08.04.2015, 17:50

\bibitem{squeezy} 
\textbf{Squeezy money - Porsche and VW} economist.com \\
\textit{http://www.economist.com/node/12523898}
\newline last used: 08.04.2015, 20:38

\bibitem{audiRinge} 
\textbf{Vier Marken - Vier Ringe} audi.com \\
\textit{http://www.audi.com/corporate/de/unternehmen/historie/unternehmen-und-marken/vier-marken-vier-ringe.html}
\newline last used: 08.04.2015, 19:08

\bibitem{audiNeuwagen} 
\textbf{Audi - Neuwagen} audi.de \\
\textit{http://www.audi.de/de/brand/de/neuwagen.html}
\newline last used: 08.04.2015, 19:20

\bibitem{vwPorscheUebernahme} 
\textbf{Die Presse - Volkswagen schließt Porsche-Übernahme ab} diepresse.com \\
\textit{http://diepresse.com/home/wirtschaft/international/1274167/Volkswagen-schliesst-PorscheUebernahme-ab}
\newline last used: 08.04.2015, 19:52

\bibitem{ScaniaMAN} 
\textbf{VW/Scania – "Der Deal macht keinen Sinn"} wirtschaftsblatt.at \\
\textit{http://wirtschaftsblatt.at/home/boerse/europa/1566768/VWScania-Der-Deal-macht-keinen-Sinn}
\newline last used: 08.04.2015, 21:02

\bibitem{adsfhgterwdsfhgf} 
\textbf{Porsche Holding Bericht} \\
\textit{http://www.bafin.de/SharedDocs/Downloads/DE/Befreiungsentscheidung/volkswagen\\\_ag\_ua.pdf?\_\_blob=publicationFile\&v=5
}
\newline last used: 09.04.2015, 12:30


\bibitem{adsfhgterwdsfhgf2} 
\textbf{Vorzugsaktien - Neue Blüte dank VW}, Frankfurter Allgemeine \\
\textit{http://www.faz.net/aktuell/finanzen/aktien/vorzugsaktien-neue-bluete-dank-vw-1894737/infografik-volkswagen-henkel-1903555.html}
\newline last used: 09.04.2015, 12:31



\end{thebibliography}
\end{document}
%img : httpmarkets.ft.comresearchMarketsTearsheetsSummarys=VOWBER vwdax.PNG
% http://www.statista.com/statistics/257660/passenger-car-sales-in-selected-countries/ salesla
% http://www.statista.com/statistics/232958/revenue-of-the-leading-car-manufacturers-worldwide/ revenu
